\chapter{Introducción específica} % Main chapter title

\label{Chapter2}


%----------------------------------------------------------------------------------------
\section{Protocolos de comununicación utilizados}

  Comenzar escribiendo que aqui se detallaran los protocolos utilizados segun el modelo OSI y hablar sobre el diagrama    

  **AGREGAR UNA IMAGEN CON LOS 3 ENTES INTERCONECTADOS BAJO EL MODELO OSI !**

        Nucleo F429ZI Board <----> Dedicated Server <----> Web Client


\subsection{Protocolos de comunicación según el Modelo OSI}

\begin{enumerate}

  \item Capa física:
      \\
NUCLEO:     LwIP utiliza los protocolos de hardware subyacentes del dispositivo, como Ethernet o Wi-Fi. Como la nucleo no viene con wi-fi, se usa ethernet.
      \\
      Los protocolos de la capa física utilizados por un servidor dedicado y/o un ordenador dependen del tipo de medio físico que se utiliza para transmitir datos a través de la red. Algunos protocolos físicos comunes incluyen par trenzado, fibra óptica y cable coaxial.
      \\ 
      Se utiliza Wi-Fi entre el servidor dedicado y el ordenador? es una posibilidad pensando en tener el servidor hosteado en un data center de Trenes Argentinos y se pueda acceder via VPN u otro medio.


  \item Capa de enlace: 
    \\
 NUCLEO:   LwIP utiliza el protocolo ARP (Address Resolution Protocol) para mapear direcciones IP a direcciones físicas (MAC). 
    \\
    server+web client: Los protocolos de la capa de enlace de datos utilizados por un computador dependen del tipo de tarjeta de interfaz de red (NIC) que se esté utilizando. Algunos protocolos comunes de la capa de enlace de datos incluyen Ethernet, Wi-Fi y Token Ring.
      

  \item Capa de red:
    \\
 NUCLEO:   En la capa de red, LwIP utiliza el protocolo IP (Internet Protocol) para enrutar los paquetes de datos a través de la red.
    \\
    El protocolo de la capa de red utilizado por un servidor dedicado es típicamente IPv4 o IPv6, que es responsable de enrutar paquetes entre diferentes dispositivos en la red.
    \\
    IPv4/6 es un protocolo? me parece que IP es el protocolo.


  \item Capa de transporte:
    \\
 NUCLEO:   LwIP admite varios protocolos de transporte, incluyendo TCP (Transmission Control Protocol) y UDP (User Datagram Protocol), que se utilizan para proporcionar una conexión confiable o no confiable, respectivamente.
    \\
  Los protocolos de la capa de transporte utilizados por un servidor dedicado incluyen TCP (Protocolo de control de transmisión) y UDP (Protocolo de datagramas de usuario), que se utilizan para proporcionar comunicación confiable y/o rápida entre diferentes aplicaciones que se ejeutan en el servidor.


  \item Capa de sesión: 
    \\
 NUCLEO:   La pila de protocolos LwIP no tiene protocolos específicos de sesión.
    \\
    el servidor dedicado y/o el ordenador ofrecen capa de sesion ?


  \item Capa de presentación: 
    \\
 NUCLEO:   La pila de protocolos LwIP no tiene protocolos específicos de presentación.
    \\
    el servidor dedicado y/o el ordenador ofrecen capa de presentacion ?


  \item Capa de aplicación: 
    \\
 NUCLEO:   En la capa de aplicación, LwIP admite varios protocolos de aplicación, incluyendo HTTP (Hypertext Transfer Protocol), FTP (File Transfer Protocol) y DNS (Domain Name System), entre otros.
    \\
    Los protocolos de la capa de aplicación utilizados por un computador dependen de las aplicaciones específicas que se utilizan en el mismo. Algunos ejemplos de protocolos de la capa de aplicación comúnmente utilizados incluyen HTTP (Protocolo de transferencia de hipertexto), FTP (Protocolo de transferencia de archivos), SMTP (Protocolo simple de transferencia de correo) y SSH (Shell seguro).
    \\
    HABLAR SOBRE MQTT entre el servidor dedicado y la nucleo 

\end{enumerate}


\subsection{LwIP Stack}

\begin{itemize}

  \item La pila de protocolos LwIP (Lightweight IP) desarrollada por STMicroelectronics para sus dispositivos STM32 también sigue el modelo OSI de siete capas.

  \item Es importante destacar que la pila de protocolos LwIP desarrollada por STM32 es altamente configurable y se puede adaptar a diferentes necesidades y requisitos de la aplicación. Además, STMicroelectronics proporciona una variedad de ejemplos y documentación para ayudar a los desarrolladores a implementar la pila de protocolos LwIP en sus dispositivos STM32.

\end{itemize}


\subsection{Servidor dedicado}


\begin{itemize}

  \item Un servidor dedicado y/o un ordenador típicamente utiliza una variedad de protocolos basados en el modelo OSI (Interconexión de Sistemas Abiertos) para admitir diferentes funcionalidades de red. 

  \item Es importante tener en cuenta que los protocolos específicos utilizados por un servidor dedicado y/o un ordenador pueden variar según el sistema operativo del servidor, la configuración del hardware y la configuración de la red.

\end{itemize}


\subsection{Protocolos por analizar si estan involucrados en algun punto de la arquitectura}


WebSocket: WebSocket is a communication protocol designed for bi-directional, real-time communication between IoT devices and a dashboard. It can be used to provide real-time updates to the dashboard and enable real-time control of IoT devices.

RESTful API: Representational State Transfer (REST) is an architectural style used for creating web services. RESTful APIs can be used to transfer data between IoT devices and a web dashboard.

GraphQL API https://hub.qovery.com/guides/tutorial/deploy-fullstack-application-composed-of-hasura-postgresql-angular



%----------------------------------------------------------------------------------------
\section{Tecnologías del \textit{frontend}}

\begin{itemize}

  \item typescript 
  
  \item react + react-dom + react-router-dom

  \item parcel 

  \item graphql
  
  \item apollo

  \item material ui


  \item jwt-decode

  \item query string

  \item WebSocket

  \item uuid
  
  \item process 

  \item tsutils

\end{itemize}



%----------------------------------------------------------------------------------------
\section{Tecnologías del \textit{backend}}

\begin{itemize}


  \item kotlin

  \item kafka: clients, streams, avro-serializer, schema-registry

  \item paho mqtt client

  \item ktor: server-netty, server-core, server-host-common, server-tests, auth, client-core, client-cio, client-serialization, client-serialization-jvm, serialization, serialization-json, network, 


  \item kgraphql + kgraphql-client

  \item apollo: runtime, api

  \item krypto

  \item PostgreSQL

  \item docker

  \item hasura

  \item mosquitto mqtt broker


\end{itemize}


%----------------------------------------------------------------------------------------
\section{Tecnologías del \textit{firmware}

\begin{itemize}

  \item C lang

  \item FreeRTOS

  \item paho mqtt client

  
\end{itemize}



%----------------------------------------------------------------------------------------
\section{Herramientas utilizadas}

\begin{itemize}

  
  \item git

  \item mqtt.fx 

  \item hive-mq broker console

  \item neovim

  \item CLion

  \item wireshark

  \item GDB o JTAG para depuración de bajo nivel

  \item brave browser (based on chromium): herramientas de desarrollo, son herramientas fundamentales para el desarrollo de frontend

  \item ST-Link de STMicroelectronics

  \item STM32CubeMX: ioc chip configuration

  \item STM32CubeProgrammer: SWV tool to monitor messages


\end{itemize}



\begin{figure}[htpb]
	\centering
	\includegraphics[scale=.3]{./Figures/word.jpeg}
	\caption{Imagen tomada de la página oficial del procesador\protect\footnotemark.}
	\label{fig:palabraIngles}
\end{figure}

\footnotetext{Imagen tomada de \url{https://goo.gl/images/i7C70w}}


